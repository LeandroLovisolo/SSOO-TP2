\documentclass[a4paper,10pt,twoside]{article}

\usepackage[top=1in, bottom=1in, left=1in, right=1in]{geometry}
\usepackage[utf8]{inputenc}
\usepackage[spanish,es-ucroman,es-noquoting]{babel}
\usepackage{setspace}
\usepackage{fancyhdr}
\usepackage{lastpage}
\usepackage{amsmath}
\usepackage{amsfonts}
\usepackage{verbatim}
\usepackage{graphicx}
\usepackage{float}
\usepackage{algpseudocode}
\usepackage{enumitem} % Provee macro \setlist
\usepackage[toc, page]{appendix}


%%%%%%%%%% Configuración de Fancyhdr - Inicio %%%%%%%%%%
\pagestyle{fancy}
\thispagestyle{fancy}
\lhead{Trabajo Práctico 2 · Sistemas Operativos}
\rhead{Delgado · Lovisolo · Petaccio}
\renewcommand{\footrulewidth}{0.4pt}
\cfoot{\thepage /\pageref{LastPage}}

\fancypagestyle{caratula} {
   \fancyhf{}
   \cfoot{\thepage /\pageref{LastPage}}
   \renewcommand{\headrulewidth}{0pt}
   \renewcommand{\footrulewidth}{0pt}
}
%%%%%%%%%% Configuración de Fancyhdr - Fin %%%%%%%%%%


%%%%%%%%%% Configuración de Algorithmic - Inicio %%%%%%%%%%
% Entorno propio para customizar la presentación del pseudocódigo
\newenvironment{pseudo}[1][]{%
    \vspace{0.5em}%
    \begin{algorithmic}%
}
{%
    \end{algorithmic}%
    \vspace{0.5em}%
}

% Cláusula return para usar así: \PReturn foo
\newcommand{\PReturn}[1]{\textbf{return} $#1$}

% Conectivos lógicos
\newcommand{\PAnd}{\textbf{and} }
\newcommand{\POr}{\textbf{or} }

% Llamada a función para usar así: \Fn{Foo}{bar, baz}.
% Produce \textsc{Foo}$(bar, baz)$.
\newcommand{\Fn}[2]{\textsc{#1}$(#2)$}

% Conectivo 'in' para usar así: \ForAll{$foo$ \In $bar$}
\newcommand{\In}{\textbf{in} }

% Conectivo 'to' para usar así: \For{$i = 1$ \In $n$}
\newcommand{\To}{\textbf{to} }

% Complejidades
\newcommand{\Ode}[1]{\hfill $O(#1)$}
%%%%%%%%%% Configuración de Algorithmic - Fin %%%%%%%%%%


%%%%%%%%%% Configuración de Appendix - Inicio %%%%%%%%%%
% Asigna la traducción de la palabra 'Appendices'.
\renewcommand{\appendixtocname}{Apéndices}
\renewcommand{\appendixpagename}{Apéndices}
%%%%%%%%%% Configuración de Appendix - Fin %%%%%%%%%%


%%%%%%%%%% Miscelánea - Inicio %%%%%%%%%%
% Evita que el documento se estire verticalmente para ocupar el espacio vacío
% en cada página.
\raggedbottom

% Deshabilita sangría en la primer línea de un párrafo.
\setlength{\parindent}{0em}

% Separación entre párrafos.
\setlength{\parskip}{0.5em}

% Separación entre elementos de listas.
\setlist{itemsep=0.5em}
%%%%%%%%%% Miscelánea - Fin %%%%%%%%%%


\begin{document}


%%%%%%%%%%%%%%%%%%%%%%%%%%%%%%%%%%%%%%%%%%%%%%%%%%%%%%%%%%%%%%%%%%%%%%%%%%%%%%%
%% Carátula                                                                  %%
%%%%%%%%%%%%%%%%%%%%%%%%%%%%%%%%%%%%%%%%%%%%%%%%%%%%%%%%%%%%%%%%%%%%%%%%%%%%%%%


\thispagestyle{caratula}

\begin{center}

\includegraphics[height=2cm]{DC.png} 
\hfill
\includegraphics[height=2cm]{UBA.jpg} 

\vspace{2cm}

Departamento de Computación,\\
Facultad de Ciencias Exactas y Naturales,\\
Universidad de Buenos Aires

\vspace{4cm}

\begin{Huge}
Trabajo Práctico 2
\end{Huge}

\vspace{0.5cm}

\begin{Large}
Sistemas Operativos
\end{Large}

\vspace{1cm}

Segundo Cuatrimestre de 2013

\vspace{4cm}

\begin{tabular}{|c|c|c|}
\hline
Apellido y Nombre & LU & E-mail\\
\hline
Alejandro Nahuel Delgado & 601/11 & nahueldelgado@gmail.com\\
Leandro Lovisolo         & 645/11 & leandro@leandro.me\\
Lautaro José Petaccio    & 443/11 & lausuper@gmail.com\\
\hline
\end{tabular}

\end{center}

\newpage


%%%%%%%%%%%%%%%%%%%%%%%%%%%%%%%%%%%%%%%%%%%%%%%%%%%%%%%%%%%%%%%%%%%%%%%%%%%%%%%
%% Índice                                                                    %%
%%%%%%%%%%%%%%%%%%%%%%%%%%%%%%%%%%%%%%%%%%%%%%%%%%%%%%%%%%%%%%%%%%%%%%%%%%%%%%%


\tableofcontents

\newpage


%%%%%%%%%%%%%%%%%%%%%%%%%%%%%%%%%%%%%%%%%%%%%%%%%%%%%%%%%%%%%%%%%%%%%%%%%%%%%%%
%% Readers-Writers Lock                                                      %%
%%%%%%%%%%%%%%%%%%%%%%%%%%%%%%%%%%%%%%%%%%%%%%%%%%%%%%%%%%%%%%%%%%%%%%%%%%%%%%%


\section{Readers-Writers Lock}

Tomamos la implementación del readers-writers lock libre de inanición de escritura presentada en el libro \textit{The Little Book of Semaphores}, versión 2.1.5, por Allen B. Downey.

Este libro utiliza la analogía de una habitación en la que pueden haber o varios lectores a la vez y ningún escritor, o un único escritor por vez y ningún lector; y un interruptor que prende o apaga la luz de la habitación para señalar si ésta está ocupada o no.

Durante la creación del lock se inicializan las siguientes variables:

\begin{pseudo}
    \State $readers \leftarrow 0$ \Comment{Cantidad de \textbf{lectores} actualmente en la habitación.}
    \State $turnstile \leftarrow$ \Fn{Semaphore}{1} \Comment{Molinete que deja pasar lectores de a uno por vez.}
    \State $room\_empty \leftarrow$ \Fn{Semaphore}{1} \Comment{Vale 1 si la habitación está vacía, o 0 en caso contrario.}
    \State $readers\_mutex \leftarrow$ \Fn{Semaphore}{1} \Comment{Mutex para acceder a la variable $readers$.}
\end{pseudo}

Las operaciones del lock quedan definidas como a continuación.

\begin{pseudo}
    \Procedure{Read-Lock}{}
        \State \Fn{Wait}{turnstile} \Comment{Paso por el molinete, un lector a la vez.}
        \State \Fn{Signal}{turnstile}
        \State
        \State \Fn{Wait}{readers\_mutex}
        \State $readers \leftarrow readers + 1$ \Comment{Aumento en 1 la cantidad de lectores.}
        \If{readers = 1} \Comment{¿Soy el primer lector en llegar?}
            \State \Fn{Wait}{room\_empty} \Comment{Espero a que se vacíe la habitación, luego entro y prendo al luz.}
        \EndIf
        \State \Fn{Signal}{readers\_mutex}
    \EndProcedure
\end{pseudo}

\begin{pseudo}
    \Procedure{Read-Unlock}{}
        \State \Fn{Wait}{readers\_mutex}
        \State $readers \leftarrow readers - 1$ \Comment{Decremento en 1 la cantidad de lectores.}
        \If{readers = 0} \Comment{¿Soy el último lector en irse?}
            \State \Fn{Signal}{room\_empty} \Comment{Apago la luz al salir.}
        \EndIf
        \State \Fn{Signal}{readers\_mutex}
    \EndProcedure
\end{pseudo}

\begin{pseudo}
    \Procedure{Write-Lock}{}
        \State \Fn{Wait}{turnstile} \Comment{Bloqueo el molinete para que no ingresen nuevos lectores.}
        \State \Fn{Wait}{room\_empty} \Comment{Espero a que se vacíe la habitación; luego entro y prendo la luz.}
    \EndProcedure
\end{pseudo}

\begin{pseudo}
    \Procedure{Write-Unlock}{}
        \State \Fn{Signal}{turnstile} \Comment{Permito el ingreso a lectores.}
        \State \Fn{Signal}{room\_empty} \Comment{Apago la luz al salir (señalo que la habitación está vacía.)}
    \EndProcedure
\end{pseudo}

Cuando llega un lector (\Fn{Read-Lock}{}), éste pasa exitosamente por el molinete (\Fn{Wait}{turnstile}) pues el semáforo $turnstile$ vale inicialmente 1 y restaura el valor original del semáforo (\Fn{Signal}{turnstile}) para que subsiguientes lectores puedan ingresar. Luego incrementa en 1 la cantidad de lectores, y si resulta que es el primer lector en ingresar, espera que se vacíe la habitación en caso de estar ocupada por algún escritor (\Fn{Wait}{room\_empty}) y prende la luz al entrar ($room\_empty$ vale 0 luego del \textsc{Wait}.)

Al retirarse (\Fn{Read-Unlock}{}), el lector decrementa en 1 la cantidad de lectores, y en caso de ser el último, apaga la luz al salir (\Fn{Signal}{room\_empty}.)

Cuando llega un escritor (\Fn{Write-Lock}{}), éste bloquea el molinete para que no ingresen nuevos lectores (\Fn{Wait}{turnstile}) y en caso de que haya algún lector en la habitación, espera hasta que se retire el último lector (\Fn{Wait}{room\_empty}) y prende la luz al entrar ($room\_empty$ vale 0 luego del \textsc{Wait}.) Al bloquear el ingreso a nuevos lectores \textbf{nos aseguramos que no ocurra inanición de escritura}, pues de lo contrario podría darse el caso de tener un número muy grande de lectores ingresando constantemente a la habitación, haciendo que ésta nunca quede vacía.

Finalmente, al retirarse un escritor (\Fn{Read-Unlock}{}), éste desbloquea el molinete para permitir el paso a los lectores (\Fn{Signal}{turnstile}) y apaga la luz al salir (\Fn{Signal}{room\_empty}). 

%%%%%%%%%%%%%%%%%%%%%%%%%%%%%%%%%%%%%%%%%%%%%%%%%%%%%%%%%%%%%%%%%%%%%%%%%%%%%%%
%% Server                                                                    %%
%%%%%%%%%%%%%%%%%%%%%%%%%%%%%%%%%%%%%%%%%%%%%%%%%%%%%%%%%%%%%%%%%%%%%%%%%%%%%%%


\section{Server}

Pendiente


%%%%%%%%%%%%%%%%%%%%%%%%%%%%%%%%%%%%%%%%%%%%%%%%%%%%%%%%%%%%%%%%%%%%%%%%%%%%%%%
%% Modelo                                                                    %%
%%%%%%%%%%%%%%%%%%%%%%%%%%%%%%%%%%%%%%%%%%%%%%%%%%%%%%%%%%%%%%%%%%%%%%%%%%%%%%%


\section{Modelo}


\subsection{Modelo::Modelo}

Pendiente


\subsection{Modelo::\char`\~Modelo}

Pendiente


\subsection{Modelo::agregarJugador}

\begin{pseudo}
\Procedure{Modelo::agregarJugador}{nombre}
	\State \Fn{rlock}{lock\_jugando}
	\If{jugando != SETUP}
		\State \Fn{runlock}{lock\_jugando}
		\State return -\texttt{ERROR\_JUEGO\_EN\_PROGRESO}
	\EndIf
	
	\State nuevoid : int $\leftarrow$ 0
	\State agregado : bool $\leftarrow$ false
	\For{nuevoid = 0 To max\_jugadores -1}
		\State \Fn{wlock}{locks\_jugadores[nuevoid]}
		\If{jugadores[nuevoid] == NULL}
			\State jugadores[nuevoid] = new Jugador(nombre)
			\State \Fn{wunlock}{locks\_jugadores[nuevoid]}
			\State agregado $\leftarrow$ true
			\State break
		\EndIf
		\State \Fn{wunlock}{locks\_jugadores[nuevoid]}
	\EndFor
	 \If{!agregado}
		\State \Fn{runlock}{lock\_jugando}
		\State \PReturn{return} -\texttt{ERROR\_MAX\_JUGADORES}
	\EndIf

	\State \Fn{wlock}{lock\_cantidad\_jugadores}
	\State cantidad\_jugadores++;
	\State \Fn{wunlock}{lock\_cantidad\_jugadores}
	
	\State \Fn{runlock}{lock\_jugando}
	\State \PReturn{nuevoid}
\EndProcedure
\end{pseudo}


\subsection{Modelo::ubicar}

\begin{pseudo}
    \Procedure{Modelo::ubicar}{t\_id, xs, ys, tamanio}
        \State \Fn{wlock}{lock\_jugando}
        \If{jugando $\neq$ \texttt{SETUP}}
            \State \Fn{wunlock}{lock\_jugando}
            \State \PReturn{- \texttt{ERROR\_JUEGO\_EN\_PROGRESO}}
        \EndIf
        \State
        \State \Fn{wlock}{locks\_jugadores[t\_id]}
        \If{jugadores[t\_id] = \texttt{NULL}}
            \State \Fn{wunlock}{locks\_jugadores[t\_id]}
            \State \Fn{wunlock}{lock\_jugando}
            \State \PReturn{- \texttt{ERROR\_JUGADOR\_INEXISTENTE}}
        \EndIf
        \State
        \State $retorno \leftarrow jugadores[t\_id].$\Fn{ubicar}{xs, ys, tamanio}
        \State
        \If{$retorno \neq$ \texttt{ERROR\_NO\_ERROR}}
            \State \Fn{borrar\_barcos}{t\_id}
        \EndIf
        \State
        \If{$retorno = $ \texttt{ERROR\_NO\_ERROR} \PAnd $jugadores[t\_id].$\Fn{listo}{}} \Comment{Si el jugador está listo}
            \State \Fn{wlock}{lock\_jugadores\_listos}
            \State $jugadores\_listos \leftarrow jugadores\_listos + 1$
            \State
            \If{$jugadores\_listos = max\_jugadores$} \Comment{Si ya están listos todos los jugadores}
                \State \Fn{\_empezar}{}
            \EndIf
            \State \Fn{wunlock}{lock\_jugadores\_listos}
        \EndIf
        \State
        \State \Fn{wunlock}{locks\_jugadores[t\_id]}
        \State \Fn{wunlock}{locks\_jugando}
        \State \PReturn{retorno}
    \EndProcedure
\end{pseudo}



\subsection{Modelo::borrar\_barcos}

Pendiente


\subsection{Modelo::empezar}

Pendiente


\subsection{Modelo::finalizar}

Pendiente


\subsection{Modelo::ack}

Pendiente


\subsection{Modelo::termino}

Pendiente


\subsection{Modelo::quitarJugador}

Pendiente


\subsection{Modelo::apuntar}

Pendiente


\subsection{Modelo::tocar}

Pendiente


\subsection{Modelo::hayEventos}

Pendiente


\subsection{Modelo::dameEvento}

Pendiente


\subsection{Modelo::actualizarJugador}

Pendiente


%%%%%%%%%%%%%%%%%%%%%%%%%%%%%%%%%%%%%%%%%%%%%%%%%%%%%%%%%%%%%%%%%%%%%%%%%%%%%%%
%% Jsonificador                                                              %%
%%%%%%%%%%%%%%%%%%%%%%%%%%%%%%%%%%%%%%%%%%%%%%%%%%%%%%%%%%%%%%%%%%%%%%%%%%%%%%%


\section{Jsonificador}


\subsection{Jsonificador::start}

Pendiente


\subsection{Jsonificador::player\_info}

Pendiente


\subsection{Jsonificador::scores}

Pendiente


\end{document}